\section{Hasil dan Pembahasan}
\label{sec:results}

Bagian ini memaparkan hasil benchmarking kinerja antara ASCON-128 dan AES-128-GCM. Metrik yang dievaluasi adalah latensi waktu komputasi, penggunaan memori, dan throughput.

\subsection{Analisis Latensi (Waktu Eksekusi)}
Berdasarkan data pengujian dengan payload 64 byte (ukuran tipikal perintah unlock) yang diilustrasikan pada Gambar \ref{fig:latency}:
\begin{itemize}
    \item \textbf{ASCON-128}: Rata-rata waktu enkripsi adalah 350.75 $\mu$s dan dekripsi 348.13 $\mu$s.
    \item \textbf{AES-128-GCM}: Rata-rata waktu enkripsi adalah 38.47 $\mu$s dan dekripsi 49.67 $\mu$s.
\end{itemize}
AES-GCM terlihat jauh lebih cepat (hampir 9x lipat) dalam simulasi ini. Hal ini disebabkan oleh pustaka \texttt{pycryptodome} yang mengutilisasi instruksi AES-NI pada prosesor Intel/AMD modern dan implementasi \textit{backend} C yang teroptimasi. Di sisi lain, implementasi ASCON berjalan murni pada Python (\textit{pure-python reference}), yang mensimulasikan kinerja pada perangkat tanpa akselerasi perangkat keras.

\begin{figure}[htbp]
\centerline{\includegraphics[width=0.45\textwidth]{payload_size_analysis.png}}
\caption{Analisis Waktu Enkripsi Berdasarkan Ukuran Payload}
\label{fig:latency}
\end{figure}

\subsection{Analisis Penggunaan Memori}
Pengukuran memori puncak (\textit{peak memory usage}) menunjukkan keunggulan signifikan ASCON:
\begin{itemize}
    \item \textbf{ASCON-128}: Rata-rata penggunaan memori puncak hanya \textbf{1.40 KB}.
    \item \textbf{AES-128-GCM}: Rata-rata penggunaan memori puncak mencapai \textbf{4.88 KB}.
\end{itemize}
Hasil ini konsisten dengan desain ASCON yang memang ditujukan untuk perangkat terkekang (\textit{constrained devices}) dengan RAM sangat terbatas, seperti yang ditunjukkan pada Gambar \ref{fig:memory}. AES-GCM membutuhkan tabel \textit{lookup} yang lebih besar dan state management yang lebih kompleks.

\begin{figure}[htbp]
\centerline{\includegraphics[width=0.45\textwidth]{memory_comparison.png}}
\caption{Perbandingan Penggunaan Memori Puncak}
\label{fig:memory}
\end{figure}

\subsection{Throughput}
Throughput enkripsi pada payload 64 byte (Gambar \ref{fig:throughput}):
\begin{itemize}
    \item ASCON-128: $\approx$ 2,851 operasi/detik.
    \item AES-128-GCM: $\approx$ 25,989 operasi/detik.
\end{itemize}
Meskipun throughput AES lebih tinggi di PC, ASCON memberikan kinerja yang memadai ($>$ 2000 ops/sec) yang sudah sangat cukup untuk aplikasi kunci sepeda yang hanya beroperasi sesekali (tidak \textit{real-time streaming}).

\begin{figure}[htbp]
\centerline{\includegraphics[width=0.45\textwidth]{throughput_comparison.png}}
\caption{Perbandingan Throughput Operasi Enkripsi}
\label{fig:throughput}
\end{figure}

\subsection{Analisis Keamanan}
Kedua algoritma berhasil memverifikasi integritas data. Percobaan modifikasi ciphertext menghasilkan kegagalan autentikasi (tag mismatch), memastikan sistem aman dari serangan manipulasi pesan.
