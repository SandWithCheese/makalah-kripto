\section{Pendahuluan}
\label{sec:introduction}

Internet of Things (IoT) telah bertransformasi menjadi elemen fundamental dalam infrastruktur teknologi modern, menghubungkan miliaran perangkat mulai dari peralatan rumah tangga hingga sensor industri. Namun, proliferasi perangkat ini menghadirkan tantangan keamanan yang signifikan. Banyak perangkat IoT, seperti kunci pintar (\textit{smart locks}), sensor medis, dan aktuator industri, beroperasi dengan sumber daya yang sangat terbatas baik dari segi daya komputasi, memori, maupun energi baterai \cite{munir2025lwc}. Standar enkripsi saat ini, Advanced Encryption Standard (AES), meskipun sangat aman dan efisien pada perangkat keras modern (desktop/server), seringkali membebani perangkat mikrokontroler murah yang tidak memiliki akselerasi perangkat keras khusus (AES-NI). Implementasi perangkat lunak AES dapat menghabiskan siklus CPU dan memori yang berlebihan, yang berdampak langsung pada latensi sistem dan umur baterai.

Menanggapi tantangan ini, National Institute of Standards and Technology (NIST) menyelenggarakan kompetisi untuk mencari standar kriptografi ringan (Lightweight Cryptography/LWC) baru. Puncaknya pada Februari 2023, NIST mengumumkan keluarga algoritma \textbf{ASCON} sebagai pemenang dan standar baru untuk perlindungan data pada perangkat terbatas. Makalah ini mengusulkan evaluasi kinerja ASCON-128 dibandingkan dengan standar AES-128-GCM dalam konteks aplikasi nyata, yaitu sistem \textit{Smart Bicycle Lock}. Kasus penggunaan ini dipilih karena mewakili skenario IoT tipikal: perangkat bertenaga baterai yang memerlukan autentikasi aman, latensi rendah untuk pengalaman pengguna, dan ketahanan terhadap serangan siber.

Penelitian ini mencakup pengembangan simulasi berbasis Python untuk memodelkan interaksi kunci sepeda pintar dan mengukur metrik kinerja kunci. Fokus utama evaluasi adalah perbandingan latensi waktu enkripsi dan dekripsi, efisiensi penggunaan memori (RAM), serta kelayakan implementasi pada sistem yang beroperasi secara \textit{real-time}. Kami membatasi pengujian pada varian ASCON-128 dan AES-128-GCM menggunakan protokol Authenticated Encryption with Associated Data (AEAD), untuk memberikan wawasan teknis yang relevan bagi perancang sistem IoT berdaya rendah.

Sistematika penulisan makalah ini disusun sebagai berikut: Bagian II memaparkan landasan teori terkait konsep Kriptografi Ringan, spesifikasi ASCON, dan mode operasi GCM. Bagian III menjelaskan metodologi penelitian, termasuk desain sistem simulasi dan skenario benchmarking. Bagian IV menyajikan analisis hasil eksperimen secara kuantitatif, membandingkan latensi, memori, dan throughput. Terakhir, Bagian V memberikan kesimpulan dari temuan penelitian ini serta saran untuk pengembangan masa depan.
