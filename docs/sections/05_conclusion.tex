\section{Kesimpulan}
\label{sec:conclusion}

\subsection{Kesimpulan}
Berdasarkan implementasi dan pengujian yang dilakukan, dapat disimpulkan bahwa:
\begin{enumerate}
    \item \textbf{ASCON-128 lebih efisien dalam penggunaan memori}, mengonsumsi hanya sekitar 1.4 KB peak RAM dibandingkan AES-GCM yang membutuhkan 4.9 KB. Ini menjadikan ASCON pilihan ideal untuk mikrokontroler low-end (seperti AVR atau ARM Cortex-M0) yang memiliki keterbatasan SRAM.
    \item \textbf{AES-128-GCM menawarkan kecepatan eksekusi yang lebih tinggi} pada perangkat yang mendukung instruksi AES-NI. Namun, pada perangkat IoT tanpa akselerasi tersebut, keunggulan ini mungkin tidak signifikan atau justru berbalik karena kompleksitas tabel AES.
    \item Implementasi sistem kunci sepeda pintar berhasil mengamankan perintah \textit{unlock} menggunakan kedua algoritma dengan mekanisme \texttt{nonce} unik untuk mencegah \textit{replay attacks}.
\end{enumerate}

\subsection{Saran}
Untuk pengembangan selanjutnya, disarankan untuk:
\begin{itemize}
    \item Melakukan pengukuran daya (\textit{power consumption benchmarking}) pada perangkat keras IoT fisik (misalnya ESP32).
    \item Mengimplementasikan versi teroptimasi C dari ASCON untuk membandingkan kinerja secara lebih adil dengan AES OpenSSL/PyCryptodome.
\end{itemize}
