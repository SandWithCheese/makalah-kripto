\section{Landasan Teori}
\label{sec:background}

\subsection{Kriptografi Ringan (Lightweight Cryptography)}
Kriptografi ringan adalah cabang kriptografi yang berfokus pada perancangan algoritma yang disesuaikan untuk lingkungan terbatas (\textit{constrained environments}), seperti sensor network, RFID tags, dan mikrokontroler embedded. Tujuannya bukan untuk memberikan keamanan yang lebih rendah dari standar konvensional, melainkan memberikan keamanan setara dengan jejak implementasi (\textit{implementation footprint}) yang lebih kecil.

\subsection{ASCON-128}
ASCON adalah keluarga algoritma \textit{Authenticated Encryption with Associated Data} (AEAD) berbasis permutasi (\textit{permutation-based}) \cite{dobraunig2021ascon}. ASCON-128 menggunakan kunci 128-bit, nonce 128-bit, dan tag 128-bit. Struktur internalnya menggunakan skema Sponge dengan permutasi 320-bit. Keunggulan utama ASCON adalah efisiensi pada perangkat lunak tanpa dukungan instruksi khusus dan ketahanan terhadap \textit{side-channel attacks}.

\subsection{AES-128-GCM}
AES (Advanced Encryption Standard) dalam mode GCM (Galois/Counter Mode) adalah standar industri untuk \textit{high-speed authenticated encryption}. AES bekerja berdasarkan operasi blok cipher 128-bit. Meskipun sangat efisien pada prosesor desktop dan server modern berkat instruksi AES-NI, implementasinya pada perangkat mikrokontroler 8-bit atau 16-bit seringkali membutuhkan kode yang besar (\texttt{code size}) dan memori tabel pencarian (\texttt{S-box}) yang signifikan.
