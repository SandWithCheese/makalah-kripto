\section{Landasan Teori}
\label{sec:background}

\subsection{Kriptografi}
Kriptografi adalah ilmu dan seni untuk menjaga keamanan pesan \cite{munir2025pengantar}. Dalam skenario komunikasi klasik yang sering digambarkan dengan pengirim (Alice) dan penerima (Bob), kriptografi bertujuan untuk melindungi informasi yang dikirimkan melalui saluran publik yang tidak aman dari pihak ketiga (Eve).

Secara umum, tujuan utama kriptografi meliputi:
\begin{itemize}
    \item \textbf{Kerahasiaan (\textit{Confidentiality})}: Menjamin bahwa informasi hanya dapat diakses oleh pihak yang berwenang.
    \item \textbf{Integritas Data (\textit{Data Integrity})}: Menjamin bahwa data tidak diubah atau dimanipulasi selama transmisi.
    \item \textbf{Autentikasi (\textit{Authentication})}: Memastikan identitas pengirim atau sumber data.
    \item \textbf{Nir-sangkal (\textit{Non-repudiation})}: Mencegah pengirim menyangkal bahwa mereka telah mengirimkan pesan tersebut.
\end{itemize}

\subsection{Kriptografi Ringan (Lightweight Cryptography)}
Kriptografi ringan (\textit{Lightweight Cryptography} atau LWC) adalah cabang riset kriptografi yang dirancang khusus untuk perangkat dengan sumber daya sangat terbatas (\textit{constrained devices}) \cite{munir2025lwc}. Berbeda dengan kriptografi konvensional yang dioptimalkan untuk kecepatan tinggi pada prosesor desktop atau server, LWC berfokus pada efisiensi implementasi.

Perangkat target untuk LWC meliputi:
\begin{itemize}
    \item \textit{Internet of Things} (IoT) \textit{devices} seperti sensor dan node jaringan kecil.
    \item \textit{Smart card} dan RFID \textit{tags}.
    \item \textit{Medical implants} (implan medis).
    \item Mikrokontroler \textit{embedded} 8-bit atau 16-bit.
\end{itemize}

Keterbatasan utama yang dihadapi perangkat-perangkat ini adalah memori yang kecil (RAM dan ROM terbatas), daya komputasi rendah, dan sumber daya energi yang terbatas (beroperasi dengan baterai kecil atau \textit{energy harvesting}). Oleh karena itu, algoritma standar seperti AES atau RSA seringkali dianggap terlalu "berat" atau boros energi untuk lingkungan ini.

\subsection{Authenticated Encryption with Associated Data (AEAD)}
\textit{Authenticated Encryption with Associated Data} (AEAD) adalah kelas algoritma kriptografi yang menyediakan kerahasiaan, integritas, dan autentikasi secara bersamaan \cite{rfc9771}. Pendekatan ini mengatasi kelemahan dari penggabungan manual antara enkripsi dan \textit{Message Authentication Code} (MAC) yang rentan terhadap kesalahan implementasi.

Berdasarkan RFC 9771, algoritma AEAD menerima empat masukan:
\begin{itemize}
    \item \textbf{Secret Key ($K$)}: Kunci rahasia yang dibagikan antar pihak.
    \item \textbf{Nonce ($N$)}: Nilai unik (\textit{Number used once}) untuk mencegah serangan ulang (\textit{replay attack}).
    \item \textbf{Associated Data ($AD$)}: Data publik yang perlu diautentikasi integritasnya namun tidak dienkripsi, seperti \textit{header} paket jaringan.
    \item \textbf{Plaintext ($P$)}: Data rahasia yang akan dienkripsi dan diautentikasi.
\end{itemize}
Keluaran dari operasi AEAD adalah \textit{Ciphertext} ($C$) dan \textit{Authentication Tag} ($T$). Proses dekripsi akan memverifikasi tag $T$ terlebih dahulu; jika verifikasi gagal, maka ciphertext tidak akan didekripsi dan dianggap tidak sah.

\subsection{Advanced Encryption Standard (AES)}
AES (\textit{Advanced Encryption Standard}) adalah standar enkripsi blok simetris yang ditetapkan oleh NIST dalam publikasi FIPS 197 \cite{fips197}. AES menggantikan DES sebagai standar global.

AES beroperasi pada blok data berukuran tetap 128-bit. Panjang kunci yang didukung adalah 128, 192, dan 256 bit, yang menentukan jumlah putaran (\textit{rounds}) transformasi: 10 putaran untuk kunci 128-bit, 12 untuk 192-bit, dan 14 untuk 256-bit.
Setiap putaran AES terdiri dari empat transformasi lapisan:
\begin{enumerate}
    \item \textbf{SubBytes}: Substitusi non-linear menggunakan S-Box untuk memberikan sifat \textit{confusion}.
    \item \textbf{ShiftRows}: Permutasi baris untuk difusi.
    \item \textbf{MixColumns}: Operasi pencampuran kolom linear untuk difusi lebih lanjut (tidak dilakukan pada putaran terakhir).
    \item \textbf{AddRoundKey}: Operasi XOR antara state dengan \textit{round key}.
\end{enumerate}

\subsection{AES-128-GCM}
Galois/Counter Mode (GCM) adalah mode operasi untuk blok cipher yang direkomendasikan oleh NIST dalam SP 800-38D untuk menyediakan layanan \textit{authenticated encryption} \cite{sp80038d}.

AES-GCM menggabungkan:
\begin{itemize}
    \item \textbf{Counter Mode (CTR)}: Untuk enkripsi, yang mengubah blok cipher menjadi stream cipher, memungkinkan paralelisasi penuh.
    \item \textbf{GMAC}: Untuk autentikasi, yang menggunakan fungsi hash universal GHASH. GHASH beroperasi pada medan Galois $GF(2^{128})$.
\end{itemize}
Meskipun GCM sangat cepat pada prosesor yang memiliki instruksi \textit{carry-less multiplication} (PCLMULQDQ) dan AES-NI, operasi perkalian pada medan Galois $GF(2^{128})$ sangat membebani mikrokontroler sederhana yang tidak memiliki dukungan perangkat keras tersebut, mengakibatkan latensi tinggi dan kode program yang besar untuk tabel pre-komputasi.

\subsection{Algoritma ASCON}
ASCON adalah algoritma AEAD yang dirancang khusus untuk memenuhi kebutuhan kriptografi ringan dan telah ditetapkan sebagai pemenang kompetisi NIST Lightweight Cryptography. Spesifikasi ASCON v1.2 \cite{ascon_round2} mendefinisikan ASCON menggunakan konstruksi \textit{Sponge}.

ASCON-128, varian utama yang direkomendasikan, memiliki karakteristik:
\begin{itemize}
    \item \textbf{State Size}: 320 bit, dibagi menjadi rate $r=64$ bit dan capacity $c=256$ bit.
    \item \textbf{Key \& Tag Size}: 128 bit.
    \item \textbf{Rounds}: $a=12$ putaran untuk inisialisasi/finalisasi, dan $b=6$ putaran untuk memproses data.
\end{itemize}

Operasi ASCON terdiri dari empat fase:
\begin{enumerate}
    \item \textbf{Initialization}: Memuat kunci dan nonce ke dalam state, diikuti permutasi $p^a$.
    \item \textbf{Associated Data Processing}: Menyerap AD blok demi blok.
    \item \textbf{Plaintext Processing}: Menyerap plaintext dan memeras ciphertext (mode \textit{duplex}).
    \item \textbf{Finalization}: Menghasilkan tag autentikasi.
\end{enumerate}
Desain ASCON hanya menggunakan operasi bitwise sederhana (AND, NOT, XOR, ROT), membuatnya sangat efisien dan kecil secara ukuran kode pada mikrokontroler apa pun, sekaligus tahan terhadap serangan \textit{side-channel}.

\subsection{Metrik Evaluasi Kinerja}
Evaluasi kinerja pada perangkat terbatas difokuskan pada tiga metrik utama:
\begin{itemize}
    \item \textbf{Latensi CPU}: Waktu siklus atau durasi (dalam detik/mikrodetik) yang dibutuhkan untuk melakukan operasi enkripsi dan dekripsi lengkap.
    \item \textbf{Penggunaan Memori (RAM)}: Jumlah memori tumpukan (\textit{stack}) maksimum yang digunakan fungsi kriptografi. Hal ini krusial karena mikrokontroler seringkali hanya memiliki RAM dalam orde kilobyte.
    \item \textbf{Ukuran Kode (ROM/Flash)}: Besarnya ruang penyimpanan program yang dibutuhkan untuk menyimpan implementasi algoritma.
\end{itemize}
